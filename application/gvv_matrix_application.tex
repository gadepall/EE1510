\documentclass[journal,12pt,twocolumn]{IEEEtran}
%
\usepackage{setspace}
\usepackage{gensymb}
\usepackage{xcolor}
\usepackage{caption}
%\usepackage{subcaption}
%\doublespacing
\singlespacing

%\usepackage{graphicx}
%\usepackage{amssymb}
%\usepackage{relsize}
\usepackage[cmex10]{amsmath}
\usepackage{mathtools}
%\usepackage{amsthm}
%\interdisplaylinepenalty=2500
%\savesymbol{iint}
%\usepackage{txfonts}
%\restoresymbol{TXF}{iint}
%\usepackage{wasysym}
\usepackage{amsthm}
\usepackage{mathrsfs}
\usepackage{txfonts}
\usepackage{stfloats}
\usepackage{cite}
\usepackage{cases}
\usepackage{subfig}
%\usepackage{xtab}
\usepackage{longtable}
\usepackage{multirow}
%\usepackage{algorithm}
%\usepackage{algpseudocode}
\usepackage{enumitem}
\usepackage{mathtools}
\usepackage{iithtlc}
%\usepackage[framemethod=tikz]{mdframed}
\usepackage{listings}


%\usepackage{stmaryrd}


%\usepackage{wasysym}
%\newcounter{MYtempeqncnt}
\DeclareMathOperator*{\Res}{Res}
%\renewcommand{\baselinestretch}{2}
\renewcommand\thesection{\arabic{section}}
\renewcommand\thesubsection{\thesection.\arabic{subsection}}
\renewcommand\thesubsubsection{\thesubsection.\arabic{subsubsection}}

\renewcommand\thesectiondis{\arabic{section}}
\renewcommand\thesubsectiondis{\thesectiondis.\arabic{subsection}}
\renewcommand\thesubsubsectiondis{\thesubsectiondis.\arabic{subsubsection}}

% correct bad hyphenation here
\hyphenation{op-tical net-works semi-conduc-tor}

\lstset{
language=Python,
frame=single, 
breaklines=true
}

%\lstset{
	%%basicstyle=\small\ttfamily\bfseries,
	%%numberstyle=\small\ttfamily,
	%language=Octave,
	%backgroundcolor=\color{white},
	%%frame=single,
	%%keywordstyle=\bfseries,
	%%breaklines=true,
	%%showstringspaces=false,
	%%xleftmargin=-10mm,
	%%aboveskip=-1mm,
	%%belowskip=0mm
%}

%\surroundwithmdframed[width=\columnwidth]{lstlisting}


\begin{document}
%

\theoremstyle{definition}
\newtheorem{theorem}{Theorem}[section]
\newtheorem{problem}{Problem}
\newtheorem{proposition}{Proposition}[section]
\newtheorem{lemma}{Lemma}[section]
\newtheorem{corollary}[theorem]{Corollary}
\newtheorem{example}{Example}[section]
\newtheorem{definition}{Definition}[section]
%\newtheorem{algorithm}{Algorithm}[section]
%\newtheorem{cor}{Corollary}
\newcommand{\BEQA}{\begin{eqnarray}}
\newcommand{\EEQA}{\end{eqnarray}}
\newcommand{\define}{\stackrel{\triangle}{=}}

\bibliographystyle{IEEEtran}
%\bibliographystyle{ieeetr}

\providecommand{\nCr}[2]{\,^{#1}C_{#2}} % nCr
\providecommand{\nPr}[2]{\,^{#1}P_{#2}} % nPr
\providecommand{\mbf}{\mathbf}
\providecommand{\pr}[1]{\ensuremath{\Pr\left(#1\right)}}
\providecommand{\qfunc}[1]{\ensuremath{Q\left(#1\right)}}
\providecommand{\sbrak}[1]{\ensuremath{{}\left[#1\right]}}
\providecommand{\lsbrak}[1]{\ensuremath{{}\left[#1\right.}}
\providecommand{\rsbrak}[1]{\ensuremath{{}\left.#1\right]}}
\providecommand{\brak}[1]{\ensuremath{\left(#1\right)}}
\providecommand{\lbrak}[1]{\ensuremath{\left(#1\right.}}
\providecommand{\rbrak}[1]{\ensuremath{\left.#1\right)}}
\providecommand{\cbrak}[1]{\ensuremath{\left\{#1\right\}}}
\providecommand{\lcbrak}[1]{\ensuremath{\left\{#1\right.}}
\providecommand{\rcbrak}[1]{\ensuremath{\left.#1\right\}}}
\theoremstyle{remark}
\newtheorem{rem}{Remark}
\newcommand{\sgn}{\mathop{\mathrm{sgn}}}
\providecommand{\abs}[1]{\left\vert#1\right\vert}
\providecommand{\res}[1]{\Res\displaylimits_{#1}} 
\providecommand{\norm}[1]{\lVert#1\rVert}
\providecommand{\mtx}[1]{\mathbf{#1}}
\providecommand{\mean}[1]{E\left[ #1 \right]}
\providecommand{\fourier}{\overset{\mathcal{F}}{ \rightleftharpoons}}
%\providecommand{\hilbert}{\overset{\mathcal{H}}{ \rightleftharpoons}}
\providecommand{\system}{\overset{\mathcal{H}}{ \longleftrightarrow}}
	%\newcommand{\solution}[2]{\textbf{Solution:}{#1}}
\newcommand{\solution}{\noindent \textbf{Solution: }}
\providecommand{\dec}[2]{\ensuremath{\overset{#1}{\underset{#2}{\gtrless}}}}
%\numberwithin{equation}{subsection}
\numberwithin{equation}{problem}
%\numberwithin{problem}{subsection}
%\numberwithin{definition}{subsection}
\makeatletter
\@addtoreset{figure}{problem}
\makeatother

\let\StandardTheFigure\thefigure
%\renewcommand{\thefigure}{\theproblem.\arabic{figure}}
\renewcommand{\thefigure}{\theproblem}


%\numberwithin{figure}{subsection}

\def\putbox#1#2#3{\makebox[0in][l]{\makebox[#1][l]{}\raisebox{\baselineskip}[0in][0in]{\raisebox{#2}[0in][0in]{#3}}}}
     \def\rightbox#1{\makebox[0in][r]{#1}}
     \def\centbox#1{\makebox[0in]{#1}}
     \def\topbox#1{\raisebox{-\baselineskip}[0in][0in]{#1}}
     \def\midbox#1{\raisebox{-0.5\baselineskip}[0in][0in]{#1}}

\vspace{3cm}

\title{ 
	\logo{
Matrix Analysis Applications
	}
}
%\title{
%	\logo{Matrix Analysis through Octave}{\begin{center}\includegraphics[scale=.24]{tlc}\end{center}}{}{HAMDSP}
%}


% paper title
% can use linebreaks \\ within to get better formatting as desired
%\title{Matrix Analysis through Octave}
%
%
% author names and IEEE memberships
% note positions of commas and nonbreaking spaces ( ~ ) LaTeX will not break
% a structure at a ~ so this keeps an author's name from being broken across
% two lines.
% use \thanks{} to gain access to the first footnote area
% a separate \thanks must be used for each paragraph as LaTeX2e's \thanks
% was not built to handle multiple paragraphs
%

%\author{Alok Ranjan Kesari$^{\dagger}$ and G V V Sharma$^{*}$ %<-this  stops a space
\author{G V V Sharma$^{*}$ %<-this  stops a space
\thanks{*The author is with the Department
of Electrical Engineering, Indian Institute of Technology, Hyderabad
502285 India e-mail:  gadepall@iith.ac.in. This work was funded by the PMMMNMTT, MHRD, Govt. of India.  All content in this manuscript is released under GNU GPL.  Free to use for  all.}% $\dagger$ The author was an intern at IIT Hyderabad in the summer of 2017 e-mail:  alok.kesari@yahoo.co.in.}% <-this % stops a space
%\thanks{J. Doe and J. Doe are with Anonymous University.}% <-this % stops a space
%\thanks{Manuscript received April 19, 2005; revised January 11, 2007.}}
}
% note the % following the last \IEEEmembership and also \thanks - 
% these prevent an unwanted space from occurring between the last author name
% and the end of the author line. i.e., if you had this:
% 
% \author{....lastname \thanks{...} \thanks{...} }
%                     ^------------^------------^----Do not want these spaces!
%
% a space would be appended to the last name and could cause every name on that
% line to be shifted left slightly. This is one of those "LaTeX things". For
% instance, "\textbf{A} \textbf{B}" will typeset as "A B" not "AB". To get
% "AB" then you have to do: "\textbf{A}\textbf{B}"
% \thanks is no different in this regard, so shield the last } of each \thanks
% that ends a line with a % and do not let a space in before the next \thanks.
% Spaces after \IEEEmembership other than the last one are OK (and needed) as
% you are supposed to have spaces between the names. For what it is worth,
% this is a minor point as most people would not even notice if the said evil
% space somehow managed to creep in.



% The paper headers
%\markboth{Journal of \LaTeX\ Class Files,~Vol.~6, No.~1, January~2007}%
%{Shell \MakeLowercase{\textit{et al.}}: Bare Demo of IEEEtran.cls for Journals}
% The only time the second header will appear is for the odd numbered pages
% after the title page when using the twoside option.
% 
% *** Note that you probably will NOT want to include the author's ***
% *** name in the headers of peer review papers.                   ***
% You can use \ifCLASSOPTIONpeerreview for conditional compilation here if
% you desire.




% If you want to put a publisher's ID mark on the page you can do it like
% this:
%\IEEEpubid{0000--0000/00\$00.00~\copyright~2007 IEEE}
% Remember, if you use this you must call \IEEEpubidadjcol in the second
% column for its text to clear the IEEEpubid mark.

\maketitle


%\tableofcontents


%\begin{abstract}
%%\boldmath
%In this letter, an algorithm for evaluating the exact analytical bit error rate  (BER)  for the piecewise linear (PL) combiner for  multiple relays is presented. Previous results were available only for upto three relays. The algorithm is unique in the sense that  the actual mathematical expressions, that are prohibitively large, need not be explicitly obtained. The diversity gain due to multiple relays is shown through plots of the analytical BER, well supported by simulations. 
%
%\end{abstract}
% IEEEtran.cls defaults to using nonbold math in the Abstract.
% This preserves the distinction between vectors and scalars. However,
% if the journal you are submitting to favors bold math in the abstract,
% then you can use LaTeX's standard command \boldmath at the very start
% of the abstract to achieve this. Many IEEE journals frown on math
% in the abstract anyway.

% Note that keywords are not normally used for peerreview papers.
%\begin{IEEEkeywords}
%Cooperative diversity, decode and forward, piecewise linear
%\end{IEEEkeywords}



% For peer review papers, you can put extra information on the cover
% page as needed:
% \ifCLASSOPTIONpeerreview
% \begin{center} \bfseries EDICS Category: 3-BBND \end{center}
% \fi
%
% For peerreview papers, this IEEEtran command inserts a page break and
% creates the second title. It will be ignored for other modes.
\IEEEpeerreviewmaketitle

\bigskip

\begin{abstract}
This manual provides some examples of matrix analysis in research.  
\end{abstract}

%\newpage
%\section{Least Squares}
%%
%%\chapter{The Optimum Receiver}

\subsection{Problem}

\begin{problem}
Sketch the vectors
%
	\begin{equation}
	\mbf{a}_1 = \brak{1,1,1}^T , \mbf{a}_2=\brak{0,1,2}^T, \mbf{b}=\brak{6,0,0}^T
	\end{equation}
	%
in the 3-D plane.
\end{problem}
%
\begin{problem}
	Find $x_1, x_2$ such that
	\begin{equation}
x_1 \mbf{a}_1 + x_2 \mbf{a}_2 = \mbf{b}
	\end{equation}
\end{problem}
%
geometrically.
\begin{problem}
Solve the matrix equation 
%
\begin{equation}
\mbf{Ax} = \mbf{b}
\end{equation}
%
where $\mbf{A} = \sbrak{\mbf{a}_1 \,\mbf{a}_2}$ using row reduction.  Comment.
\end{problem}
\subsection{Solution using Octave}
\begin{problem}
	Type the following program in octave and comment on the output for different values of $\mbf{x}$
	\begin{verbatim}
%Code written by GVV Sharma March 30, 2016
%Released under GNU GPL.  Free to use for anything.

%This program compares the norm defined for the least-squares solution
%for the correct solution vs other data points.
%You will find that the metric is the smallest for the correct value.

clear;
close;

A = [1 0; 1 1; 1 2]; %The input matrix
b = [6;0;0]; %The output vector

P = inv(A'*A)*A';%pseudoinverse

x_ls = P*b; %The least squares solution

x = [5;-5]; %Any random input

exact_ls_metric = norm(b-A*x_ls)^2 %The metric for actual soltuion
random_ls_metric = norm(b-A*x)^2 %metric for a random value of x
	
	\end{verbatim}

\end{problem}
%
\begin{problem}
	Type the following code in Octave and observe the output.
\begin{verbatim}
%Code written by GVV Sharma March 31, 2016
%Released under GNU GPL.  Free to use for anything.


%This program plots the least squares metric for a range of
%vectors x in the mesh with vertices (-10,-10),(-10,10),(10,-10)
%%and (10,10)

%The result is a 3-D mesh.  The theoretical minimum is (5,-3)
%Values obtained through the following program are close to the 
%theoreticl solution


clear;
close;

A = [1 0; 1 1; 1 2]; %The input matrix
b = [6;0;0]; %The output vector


x1 = linspace(-10,10,50); %generating points in x-axis
x2 = linspace(-10,10,50);  %generating points in y-axis

[xx, yy] = meshgrid(x1,x2);

ffun = @(x,y) norm(b-A*[x;y])^2;

f = arrayfun(ffun,xx,yy);

mesh(xx,yy,f)

[M I] = min(f(:)); %vectorize the 50 x 50 matrix f, find min
%M = min value , I is the index of the f_min

[I_r I_c] = ind2sub(size(f),I); %Get the row, col index of f_min


%The least square solution
xx(I_r,I_c) 
yy(I_r,I_c)
%The minimum value of metric
M

\end{verbatim}
\end{problem}
%
\begin{problem}
	Compare the results obtained by typing the following code with the results in the previous problem.
\begin{verbatim}
%Code written by GVV Sharma March 31, 2016
%Released under GNU GPL.  Free to use for anything.


%This program finds the theoretical least squares solution using 
%SVD 

clear;
close;

A = [1 0; 1 1; 1 2]; %The input matrix
b = [6;0;0]; %The output vector


[U S V] = svd(A); % Computing the SVD of A

temp_S = 1./diag(S); %inverting the diagonal values of S

Splus = [diag(temp_S) zeros(2,1)]; %inverse transpose of S

Aplus = V*Splus*U'; %The Moore-Penrose pseudo-inverse

Aplus*b %least squares solution.

\end{verbatim}
\end{problem}
%
\begin{problem}
	Type the following code in Octave and run.  Comment.
\begin{verbatim}
%Code written by GVV Sharma March 31, 2016
%Released under GNU GPL.  Free to use for anything.


%This program finds the SVD for the matrix A
%Involves eigenvalue decomposition as well as 
%QR factorization (Gram-Schmidt Orthogonalization)

%Note that the columns of U and V are interchanged
%when compared with the U and V matrices obtained 
%using the builtin SVD command.


clear;
close;

A = [1 0; 1 1; 1 2]; %The input matrix
b = [6;0;0]; %The output vector


[Pv,Dv] = eig(A'*A);%Eigenvalue decomposition of A'*A
[Pu,Du] = eig(A*A');%Eigenvalue decomposition of A*A'


Stemp = sqrt(Dv); %singular values of A 
[V,Rv] = qr(Pv); %V
[U,Ru] = qr(Pu);  %U

\end{verbatim}
\end{problem}
%
	Let 
	\begin{equation}
	g(\mbf{x}) = \norm{\mbf{b} -\mbf{Ax}}^2
	\end{equation}
	
%
\begin{problem}
		Using calculus, minimize $g(\mbf{x})$.
\end{problem}
%
\begin{problem}
Find $(A^TA)^{-1}A^Tb$
\end{problem}


%%
%
%%\newpage
%\section{Matrix Analysis}
%Verify your results through Python, wherever possible.

\subsection{Eigenvalues and Eigenvectors}
For any square matrix $\mbf{G}$, if
%
\begin{equation}
\mbf{G}\mbf{x} = \lambda \mbf{x},
\end{equation}
%
$\lambda$ is known as the {\em eigenvalue} and $\mbf{x}$ is the corresponding {\em eigenvector}.

Let
%
\begin{equation}
\mbf{G} = \begin{pmatrix}
1 & 1\\-2 & 4
\end{pmatrix}
\end{equation}
\begin{problem}
Show that the eigenvalues of $\mbf{G}$ are obtained by solving the equation
%
\begin{equation}
\label{char_equation}
f\brak{\lambda}=\abs{\lambda \mbf{I}- G} = 0
\end{equation}
%
\end{problem}
Note that \eqref{char_equation} is known as the {\em characteristic equation}.  $f\brak{\lambda}$ is known as the characteristic polynomial.

\begin{problem}
	Obtain the eigenvalues and eigenvectors of $\mbf{G}$.
\end{problem}
\begin{problem}
	Find $f(\mbf{G})$.  This is known as the {\em Cayley-Hamilton Theorem}.
\end{problem}


\begin{problem}
	Stack the eigenvalues of $\mbf{G}$ in a diagonal matrix $\mbf{\Lambda}$ and the corresponding eigenvectors in a matrix $\mbf{F}$.  Find $\mbf{F}\mbf{\Lambda}\mbf{F}^{-1}$.  This is known as {\em Eigenvalue Decomposition}
\end{problem}
%

\subsection{Symmetric Matrices}
Let 
%
\begin{equation}
\mbf{C} = \begin{pmatrix}
37 & 9 \\9 & 13
\end{pmatrix}
\end{equation}
%
Note that $\mbf{C} = \mbf{C}^T$.  Such matrices are known as {\em symmetric matrices}.

\begin{problem}
Find $\mbf{P}$ such that  $\mbf{C} = \mbf{P}\mbf{D}\mbf{P}^{-1}$, where $\mbf{D}$ is a diagonal matrix.  
\end{problem}

\begin{problem}
	Find $\mbf{P}\mbf{P}^T$ and $\mbf{P}^T\mbf{P}$.  $\mbf{P}$ is known as an {\em orthogonal matrix}.
\end{problem}
Let
\begin{equation}
\mbf{B} =
\begin{pmatrix}
4 & 11 & 14
\\
8 & 7 & -2
\end{pmatrix}
\end{equation}
%
\begin{problem}
Find $\mbf{B}^T\mbf{B}$ and $\mbf{B}\mbf{B}^T$
\end{problem}

Note that $\mbf{C} = \frac{1}{9}\brak{\mbf{B}\mbf{B}^T}$.  

\begin{problem}
	Obtain the eigenvalues and eigenvectors of $\mbf{B}^T\mbf{B}$
\end{problem}
%
\begin{problem}
	Verify eigenvalue decomposition and Cayley-Hamilton theorem for $\mbf{B}^T\mbf{B}$.
\end{problem}

\subsection{Orthogonality}
Let $\mbf{v}_1,\mbf{v}_2$ be the columns of $\mbf{C}$.

\begin{problem}
	Obtain $\mbf{u}_1,\mbf{u}_2$ from $\mbf{v}_1,\mbf{v}_2$ through the following equations. 
	%
\begin{align}
\mbf{u}_1&= \frac{\mbf{v}_1}{\norm{\mbf{v}_1}}
\\
\hat{\mbf{u}}_2 &= \mbf{v}_2 - \brak{\mbf{v}_2,\mbf{u}_1}\mbf{u}_1
\\
\mbf{u}_2 &= \frac{\hat{\mbf{u}}_2}{\norm{\hat{\mbf{u}}_2}}
\end{align}
	%
	This procedure is known as Gram-Schmidt orthogonalization.
\end{problem}

\begin{problem}
Stack the vectors $\mbf{u}_1,\mbf{u}_2$ in columns to obtain the matrix $\mbf{Q}$.  Show that $\mbf{Q}$ is orthogonal.  
\end{problem}

\begin{problem}
	From the Gram=Schmidt process, show that $\mbf{C}=\mbf{Q}\mbf{R}$, where $\mbf{R}$ is an upper triangular matrix.  This is known as the $\mbf{Q}-\mbf{R}$ decomposition.  
\end{problem}


\subsection{Singular Value Decomposition}

\begin{problem}
	Find an orthonormal basis for $\mbf{B}^T\mbf{B}$ comprising of the eigenvectors.  Stack these orthonormal eigenvectors in a matrix $\mbf{V}$. This is known as {\em Orthogonal Diagonalization}.  
\end{problem}
\begin{problem}
	Find the singular values of $\mbf{B}^T\mbf{B}$.  The singular values are obtained by taking the square roots of its eigenvalues.  
\end{problem}
\begin{problem}
	Stack the singular values of $\mbf{B}^T\mbf{B}$ diagonally to obtain a matrix $\mbf{\Sigma}$.
\end{problem}

\begin{problem}
	Obtain the matrix $\mbf{B}\mbf{V}$.  Verify if the columns of this matrix are orthogonal.
\end{problem}

\begin{problem}
	Extend the columns of $\mbf{B}\mbf{V}$ if necessary, to obtain an orthogonal matrix $\mbf{U}$.
\end{problem}

\begin{problem}
	Find $\mbf{U}\mbf{\Sigma}\mbf{V}^T$.  Comment.
\end{problem}

\subsection{Quadratic Forms}

%\begin{problem}
%	Type the following in Python and interpret the output.  $\theta = \mbf{x}^T\mbf{C}\mbf{x}$ is known as the {\em Quadratic Form} for $\mbf{C}$. $\theta$ is defined for a {\em Symmetric Matrix}.
%	\begin{verbatim}
%	%Code written by GVV Sharma April 10, 2016
%	%Released under GNU GPL.  Free to use for anything.
%	
%	%This program plots the quadratic form for a range of
%	%vectors x in the mesh with vertices (-10,-10),(-10,10),(10,-10)
%	%%and (10,10)
%	
%	%The result is a 3-D mesh.  
%	%The quadratic form in terms of the eigenvalues of the
%	%symmetric matrix is explored through this program.
%	
%	
%	clear;
%	close;
%	
%	C = [37 9; 9 13];
%	[P lambda] = eig(C);
%	
%	x1 = linspace(-10,10,50); %generating points in x-axis
%	x2 = linspace(-10,10,50);  %generating points in y-axis
%	
%	[xx, yy] = meshgrid(x1,x2);
%	
%	ffun = @(x,y) [x y]*C*[x;y];
%	
%	f = arrayfun(ffun,xx,yy);
%	
%	mesh(xx,yy,f)
%	
%	[M I] = min(f(:)); %vectorize the 50 x 50 matrix f, find min
%	%M = min value , I is the index of the f_min
%	
%	[I_r I_c] = ind2sub(size(f),I); %Get the row, col index of f_min
%	
%	%The minimum value of the quadratic form
%	M
%	%Verifying the eigenvalue relation
%	x_hat = [xx(I_r,I_c);  yy(I_r,I_c)]
%	x_hat'*C*x_hat
%	z = P*[xx(I_r,I_c);  yy(I_r,I_c)]
%	z'*lambda*z
%	
%	\end{verbatim}
%\end{problem}

\begin{problem}
$\theta = \mbf{x}^T\mbf{C}\mbf{x}$ is known as the {\em Quadratic Form} for $\mbf{C}$. $\theta$ is defined for a {\em Symmetric Matrix}.	A matrix for which the quadratic form is always positive is known as a {\em positive definite} matrix.  Is $\mbf{C}$  positive definite?
\end{problem}
\begin{problem}
	Find out the relation between positive definiteness and the eigenvalues of a symmetric matrix.
\end{problem}

\begin{problem}
	Find the minimum and maximum values of $\theta =\mbf{x}^T\mbf{C}\mbf{x}$, if $\norm{\mbf{x}} =1$.  
\end{problem}

%\begin{problem}
%	Assuming that the eigenvectors in $\mbf{P}$ are stacked in decreasing order of the eigenvalues, verify if the first column of $\mbf{P}$ yields a maximum for $\theta$.
%\end{problem}
%\vspace{2cm}

%
%%\newpage
%\section{Application in Research}
%\subsection{Orthogonal Modulation}

%Consider the matrix equation
%
%\begin{equation}
%\label{ls_simple}
%\mbf{r}_i = \mathbf{H}\mbf}_i %+\mbf{n}
%\end{equation}
%
\begin{problem}
Let 
%
\begin{equation}
r = \sum_{j=1}^{2}h_{j}c_j 
\end{equation}
%
Express the above as a matrix  equation. Note that $r$ is a scalar.
\end{problem}
\begin{problem}
Let
%
\begin{equation}
r_i = \sum_{j=1}^{2}h_{ij}c_j, \quad i = 1,2.
\end{equation}
%
Express the above as the matrix equation
%
\begin{equation}
\label{two}
\mbf{r} = \mbf{H}\mbf{c}
\end{equation}
%
List the entries of each matrix/vector in \eqref{two}.
	\end{problem}
	\begin{problem}
If
%
\begin{equation}
r_i = \sum_{j=1}^{N}h_{ij}c_j, \quad i = 1,2 \dots M,
\end{equation}
%
what is the dimension of the matrix $\mbf{H}$ in the matrix equation? 
\end{problem}
\begin{problem}
Let
%
\begin{equation}
\mbf{r}^t = \mbf{h}^t\mbf{C}
\end{equation}
%
where $\mbf{r}$ is $L\times 1$ vector and  $C$ is an $N \times L$ matrix.  Find the least squares estimate for $\mbf{h}$. What is the size of $\mbf{h}$?
\end{problem}
\begin{problem}
Now consider the matrix equation
%
\begin{equation}
\mbf{R} = \mbf{H}\mbf{C} 
\end{equation}
%
where $\mbf{R}$ is $M\times L$, $\mbf{H}$ is $M \times N$ and $\mbf{C}$ is $N \times L$.  Find the least squares estimate of $\mbf{H}$.
\end{problem}
\begin{problem}
Let
%
\begin{equation}
D = x_1^2 - x_2^2
\end{equation}
$D$ can be expressed in quadratic form as $D = \mbf{x}^{t}Q\mbf{x}$, where
$\mbf{x} = \brak{x_1,x_2}^{t}$.  Find $Q$.
\end{problem}
\begin{problem}
Find the determinant and eigenvalues of 
%
\begin{equation}
\mbf{A}=\begin{pmatrix}
1 & 2 \\
3 & 2
\end{pmatrix}
\end{equation}
%
\end{problem}
\begin{problem}
Find the determinant and eigenvalues of $\mbf{A} \otimes \mbf{I}$, where $\mbf{I}$ is the $2\times 2$ identity matrix.  Comment.
%
\end{problem}
\begin{problem}
Find the eigenvalues of $I-kA$, without explicitly calculating them. $k$ is a constant.
\end{problem}
Consider the matrix
%
\begin{equation}
\mbf{S} = 
\begin{pmatrix}
s_1 & s_2 \\
-s_2^{*} & s_1^{*}
\end{pmatrix}
%
\end{equation} 
%
where $*$ represents the conjugate of a scalar and conjugate transpose of a vector.
\begin{problem}
Find $SS^{*}$.  Comment.
\end{problem}
\begin{problem}
Express
%
\begin{equation}
\label{alamouti}
\begin{split}
r_1 &= h_1s_1+h_2s_2 \\
r_2 &= -h_1s_2^{*}+h_2s_1^{*} \\
\end{split}
\end{equation}
%
as a matrix equation.
\end{problem}
\begin{problem}
Solve for  $s_1$ and $s_2$ in \eqref{alamouti} using matrices.
\end{problem}

{\em The problems in this chapter were framed using \cite{parul} and \cite{alamouti}}.  The primary reference for this manual is \cite{davidlay}.





%\begin{center}
% \scalebox{1}{%
%\normalsize
%\parbox{6.57292in}{%
%\includegraphics[scale=0.7]{quiz1_prob1.eps} \\
% translate x=960 y=544 scale 0.38
%\putbox{3.0in}{2.5in}{1.20}{$V$}%
%} % close 'parbox'
%} % close 'scalebox'
%\vspace{-\baselineskip} % this is not necessary, but looks better
%\end{center}
%






\bibliography{IEEEabrv,gvv_matrix}

%Verify your results through Python, wherever possible.

\subsection{Eigenvalues and Eigenvectors}
For any square matrix $\mbf{G}$, if
%
\begin{equation}
\mbf{G}\mbf{x} = \lambda \mbf{x},
\end{equation}
%
$\lambda$ is known as the {\em eigenvalue} and $\mbf{x}$ is the corresponding {\em eigenvector}.

Let
%
\begin{equation}
\mbf{G} = \begin{pmatrix}
1 & 1\\-2 & 4
\end{pmatrix}
\end{equation}
\begin{problem}
Show that the eigenvalues of $\mbf{G}$ are obtained by solving the equation
%
\begin{equation}
\label{char_equation}
f\brak{\lambda}=\abs{\lambda \mbf{I}- G} = 0
\end{equation}
%
\end{problem}
Note that \eqref{char_equation} is known as the {\em characteristic equation}.  $f\brak{\lambda}$ is known as the characteristic polynomial.

\begin{problem}
	Obtain the eigenvalues and eigenvectors of $\mbf{G}$.
\end{problem}
\begin{problem}
	Find $f(\mbf{G})$.  This is known as the {\em Cayley-Hamilton Theorem}.
\end{problem}


\begin{problem}
	Stack the eigenvalues of $\mbf{G}$ in a diagonal matrix $\mbf{\Lambda}$ and the corresponding eigenvectors in a matrix $\mbf{F}$.  Find $\mbf{F}\mbf{\Lambda}\mbf{F}^{-1}$.  This is known as {\em Eigenvalue Decomposition}
\end{problem}
%

\subsection{Symmetric Matrices}
Let 
%
\begin{equation}
\mbf{C} = \begin{pmatrix}
37 & 9 \\9 & 13
\end{pmatrix}
\end{equation}
%
Note that $\mbf{C} = \mbf{C}^T$.  Such matrices are known as {\em symmetric matrices}.

\begin{problem}
Find $\mbf{P}$ such that  $\mbf{C} = \mbf{P}\mbf{D}\mbf{P}^{-1}$, where $\mbf{D}$ is a diagonal matrix.  
\end{problem}

\begin{problem}
	Find $\mbf{P}\mbf{P}^T$ and $\mbf{P}^T\mbf{P}$.  $\mbf{P}$ is known as an {\em orthogonal matrix}.
\end{problem}
Let
\begin{equation}
\mbf{B} =
\begin{pmatrix}
4 & 11 & 14
\\
8 & 7 & -2
\end{pmatrix}
\end{equation}
%
\begin{problem}
Find $\mbf{B}^T\mbf{B}$ and $\mbf{B}\mbf{B}^T$
\end{problem}

Note that $\mbf{C} = \frac{1}{9}\brak{\mbf{B}\mbf{B}^T}$.  

\begin{problem}
	Obtain the eigenvalues and eigenvectors of $\mbf{B}^T\mbf{B}$
\end{problem}
%
\begin{problem}
	Verify eigenvalue decomposition and Cayley-Hamilton theorem for $\mbf{B}^T\mbf{B}$.
\end{problem}

\subsection{Orthogonality}
Let $\mbf{v}_1,\mbf{v}_2$ be the columns of $\mbf{C}$.

\begin{problem}
	Obtain $\mbf{u}_1,\mbf{u}_2$ from $\mbf{v}_1,\mbf{v}_2$ through the following equations. 
	%
\begin{align}
\mbf{u}_1&= \frac{\mbf{v}_1}{\norm{\mbf{v}_1}}
\\
\hat{\mbf{u}}_2 &= \mbf{v}_2 - \brak{\mbf{v}_2,\mbf{u}_1}\mbf{u}_1
\\
\mbf{u}_2 &= \frac{\hat{\mbf{u}}_2}{\norm{\hat{\mbf{u}}_2}}
\end{align}
	%
	This procedure is known as Gram-Schmidt orthogonalization.
\end{problem}

\begin{problem}
Stack the vectors $\mbf{u}_1,\mbf{u}_2$ in columns to obtain the matrix $\mbf{Q}$.  Show that $\mbf{Q}$ is orthogonal.  
\end{problem}

\begin{problem}
	From the Gram=Schmidt process, show that $\mbf{C}=\mbf{Q}\mbf{R}$, where $\mbf{R}$ is an upper triangular matrix.  This is known as the $\mbf{Q}-\mbf{R}$ decomposition.  
\end{problem}


\subsection{Singular Value Decomposition}

\begin{problem}
	Find an orthonormal basis for $\mbf{B}^T\mbf{B}$ comprising of the eigenvectors.  Stack these orthonormal eigenvectors in a matrix $\mbf{V}$. This is known as {\em Orthogonal Diagonalization}.  
\end{problem}
\begin{problem}
	Find the singular values of $\mbf{B}^T\mbf{B}$.  The singular values are obtained by taking the square roots of its eigenvalues.  
\end{problem}
\begin{problem}
	Stack the singular values of $\mbf{B}^T\mbf{B}$ diagonally to obtain a matrix $\mbf{\Sigma}$.
\end{problem}

\begin{problem}
	Obtain the matrix $\mbf{B}\mbf{V}$.  Verify if the columns of this matrix are orthogonal.
\end{problem}

\begin{problem}
	Extend the columns of $\mbf{B}\mbf{V}$ if necessary, to obtain an orthogonal matrix $\mbf{U}$.
\end{problem}

\begin{problem}
	Find $\mbf{U}\mbf{\Sigma}\mbf{V}^T$.  Comment.
\end{problem}

\subsection{Quadratic Forms}

%\begin{problem}
%	Type the following in Python and interpret the output.  $\theta = \mbf{x}^T\mbf{C}\mbf{x}$ is known as the {\em Quadratic Form} for $\mbf{C}$. $\theta$ is defined for a {\em Symmetric Matrix}.
%	\begin{verbatim}
%	%Code written by GVV Sharma April 10, 2016
%	%Released under GNU GPL.  Free to use for anything.
%	
%	%This program plots the quadratic form for a range of
%	%vectors x in the mesh with vertices (-10,-10),(-10,10),(10,-10)
%	%%and (10,10)
%	
%	%The result is a 3-D mesh.  
%	%The quadratic form in terms of the eigenvalues of the
%	%symmetric matrix is explored through this program.
%	
%	
%	clear;
%	close;
%	
%	C = [37 9; 9 13];
%	[P lambda] = eig(C);
%	
%	x1 = linspace(-10,10,50); %generating points in x-axis
%	x2 = linspace(-10,10,50);  %generating points in y-axis
%	
%	[xx, yy] = meshgrid(x1,x2);
%	
%	ffun = @(x,y) [x y]*C*[x;y];
%	
%	f = arrayfun(ffun,xx,yy);
%	
%	mesh(xx,yy,f)
%	
%	[M I] = min(f(:)); %vectorize the 50 x 50 matrix f, find min
%	%M = min value , I is the index of the f_min
%	
%	[I_r I_c] = ind2sub(size(f),I); %Get the row, col index of f_min
%	
%	%The minimum value of the quadratic form
%	M
%	%Verifying the eigenvalue relation
%	x_hat = [xx(I_r,I_c);  yy(I_r,I_c)]
%	x_hat'*C*x_hat
%	z = P*[xx(I_r,I_c);  yy(I_r,I_c)]
%	z'*lambda*z
%	
%	\end{verbatim}
%\end{problem}

\begin{problem}
$\theta = \mbf{x}^T\mbf{C}\mbf{x}$ is known as the {\em Quadratic Form} for $\mbf{C}$. $\theta$ is defined for a {\em Symmetric Matrix}.	A matrix for which the quadratic form is always positive is known as a {\em positive definite} matrix.  Is $\mbf{C}$  positive definite?
\end{problem}
\begin{problem}
	Find out the relation between positive definiteness and the eigenvalues of a symmetric matrix.
\end{problem}

\begin{problem}
	Find the minimum and maximum values of $\theta =\mbf{x}^T\mbf{C}\mbf{x}$, if $\norm{\mbf{x}} =1$.  
\end{problem}

%\begin{problem}
%	Assuming that the eigenvectors in $\mbf{P}$ are stacked in decreasing order of the eigenvalues, verify if the first column of $\mbf{P}$ yields a maximum for $\theta$.
%\end{problem}
%\vspace{2cm}
 
%%
%\newpage
%\section{$M$-ary Modulation}
%\input{chapter3} 
%
%\newpage
%\section{BER in Rayleigh Flat Slowly Fading Channels}
%\input{chapter4} 

\end{document}


