%\chapter{The Optimum Receiver}

\subsection{Problem}

\begin{problem}
Sketch the vectors
%
	\begin{equation}
	\mbf{a}_1 = \brak{1,1,1}^T , \mbf{a}_2=\brak{0,1,2}^T, \mbf{b}=\brak{6,0,0}^T
	\end{equation}
	%
in the 3-D plane.
\end{problem}
%
\begin{problem}
	Find $x_1, x_2$ such that
	\begin{equation}
x_1 \mbf{a}_1 + x_2 \mbf{a}_2 = \mbf{b}
	\end{equation}
\end{problem}
%
geometrically.
\begin{problem}
Solve the matrix equation 
%
\begin{equation}
\mbf{Ax} = \mbf{b}
\end{equation}
%
where $\mbf{A} = \sbrak{\mbf{a}_1 \,\mbf{a}_2}$ using row reduction.  Comment.
\end{problem}
\subsection{Solution using Octave}
\begin{problem}
	Type the following program in octave and comment on the output for different values of $\mbf{x}$
	\begin{verbatim}
%Code written by GVV Sharma March 30, 2016
%Released under GNU GPL.  Free to use for anything.

%This program compares the norm defined for the least-squares solution
%for the correct solution vs other data points.
%You will find that the metric is the smallest for the correct value.

clear;
close;

A = [1 0; 1 1; 1 2]; %The input matrix
b = [6;0;0]; %The output vector

P = inv(A'*A)*A';%pseudoinverse

x_ls = P*b; %The least squares solution

x = [5;-5]; %Any random input

exact_ls_metric = norm(b-A*x_ls)^2 %The metric for actual soltuion
random_ls_metric = norm(b-A*x)^2 %metric for a random value of x
	
	\end{verbatim}

\end{problem}
%
\begin{problem}
	Type the following code in Octave and observe the output.
\begin{verbatim}
%Code written by GVV Sharma March 31, 2016
%Released under GNU GPL.  Free to use for anything.


%This program plots the least squares metric for a range of
%vectors x in the mesh with vertices (-10,-10),(-10,10),(10,-10)
%%and (10,10)

%The result is a 3-D mesh.  The theoretical minimum is (5,-3)
%Values obtained through the following program are close to the 
%theoreticl solution


clear;
close;

A = [1 0; 1 1; 1 2]; %The input matrix
b = [6;0;0]; %The output vector


x1 = linspace(-10,10,50); %generating points in x-axis
x2 = linspace(-10,10,50);  %generating points in y-axis

[xx, yy] = meshgrid(x1,x2);

ffun = @(x,y) norm(b-A*[x;y])^2;

f = arrayfun(ffun,xx,yy);

mesh(xx,yy,f)

[M I] = min(f(:)); %vectorize the 50 x 50 matrix f, find min
%M = min value , I is the index of the f_min

[I_r I_c] = ind2sub(size(f),I); %Get the row, col index of f_min


%The least square solution
xx(I_r,I_c) 
yy(I_r,I_c)
%The minimum value of metric
M

\end{verbatim}
\end{problem}
%
\begin{problem}
	Compare the results obtained by typing the following code with the results in the previous problem.
\begin{verbatim}
%Code written by GVV Sharma March 31, 2016
%Released under GNU GPL.  Free to use for anything.


%This program finds the theoretical least squares solution using 
%SVD 

clear;
close;

A = [1 0; 1 1; 1 2]; %The input matrix
b = [6;0;0]; %The output vector


[U S V] = svd(A); % Computing the SVD of A

temp_S = 1./diag(S); %inverting the diagonal values of S

Splus = [diag(temp_S) zeros(2,1)]; %inverse transpose of S

Aplus = V*Splus*U'; %The Moore-Penrose pseudo-inverse

Aplus*b %least squares solution.

\end{verbatim}
\end{problem}
%
\begin{problem}
	Type the following code in Octave and run.  Comment.
\begin{verbatim}
%Code written by GVV Sharma March 31, 2016
%Released under GNU GPL.  Free to use for anything.


%This program finds the SVD for the matrix A
%Involves eigenvalue decomposition as well as 
%QR factorization (Gram-Schmidt Orthogonalization)

%Note that the columns of U and V are interchanged
%when compared with the U and V matrices obtained 
%using the builtin SVD command.


clear;
close;

A = [1 0; 1 1; 1 2]; %The input matrix
b = [6;0;0]; %The output vector


[Pv,Dv] = eig(A'*A);%Eigenvalue decomposition of A'*A
[Pu,Du] = eig(A*A');%Eigenvalue decomposition of A*A'


Stemp = sqrt(Dv); %singular values of A 
[V,Rv] = qr(Pv); %V
[U,Ru] = qr(Pu);  %U

\end{verbatim}
\end{problem}
%
	Let 
	\begin{equation}
	g(\mbf{x}) = \norm{\mbf{b} -\mbf{Ax}}^2
	\end{equation}
	
%
\begin{problem}
		Using calculus, minimize $g(\mbf{x})$.
\end{problem}
%
\begin{problem}
Find $(A^TA)^{-1}A^Tb$
\end{problem}

